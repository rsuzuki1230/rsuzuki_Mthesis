%%%%%%%%%%%%%%%%%%%%%%%%%%%%%%%%%%%%%%%%%%%%%%%%%%%%%%%%
%%%%%%%%             Style  Setting             %%%%%%%%
% $B%U%)%s%H(B: 12point ($B:GBg(B), $BJRLL0u:~(B
\documentclass[a4j,12pt,openbib,oneside]{jreport}

%%%%%%%%%%%%%%%%%%%%%%%%%%%%%%%%%%%%%%%%%%%%%%%%%%%%%%%%
%%%%%%%%             Package Include            %%%%%%%%
\usepackage{ascmac}
\usepackage{tabularx}
\usepackage[dvipdfmx]{graphicx}
\usepackage{amssymb}
\usepackage{amsmath}
\usepackage{mathrsfs}
\usepackage{Dennou6}            % $BEEG>%9%?%$%k(B ver 6
\usepackage{bm}
\usepackage{framed}
\usepackage[dvipdfmx]{color}
\usepackage{empheq}
\usepackage{comment}
\usepackage{fancybox}
\usepackage{enumitem}
\usepackage{mathtools}
\usepackage{listliketab}
\usepackage[stable]{footmisc}
\usepackage{setspace}
\usepackage[dvipdfmx]{hyperref}
\usepackage{pxjahyper} % $BF|K\8lBP1~(B ($B;H$o$J$$$H$b$/$8ItJ,$,J8(B>$B;z2=$1$9$k(B)
\usepackage{lscape}    % $BI=$r2#$K(B
\usepackage{url}
\usepackage[numbers,sort]{natbib}
%\usepackage[biblabel]{cite} % natbib $B$H>WFM$9$k2DG=@-$,$"$k$N$G;H$o$J$$$3$H$K$9$k(B
\usepackage{remreset}
\usepackage{subfigure}
\usepackage{here} % $B6/@)E*$K2hA|0LCV$r;XDj(B
%%%%%%%%%%%%%%%%%%%%%%%%%%%%%%%%%%%%%%%%%%%%%%%%%%%%%%%%
%%%%%%%%            PageStyle Setting           %%%%%%%%
%\pagestyle{Dheadings}
\pagestyle{DAmyheadings}
%%%%%%%%%%%%%%%%%%%%%%%%%%%%%%%%%%%%%%%%%%%%%%%%%%%%%%%%
%%%%%%%%        Title and Auther Setting        %%%%%%%%
%%
%%  [ ] $B$O%X%C%@$K=q$-=P$5$l$k(B.
%%  { } $B$OI=Bj(B (\maketitle) $B$K=q$-=P$5$l$k(B.

%% $B2~CJMn;~$N6u9T@_Dj(B
%\Dparskip      % $B2~CJMn;~$K0l9T6u9T$rF~$l$k(B
\Dnoparskip    % $B2~CJMn;~$K0l9T6u9T$rF~$l$J$$(B

%% $B2~CJMn;~$N%$%s%G%s%H@_Dj(B
\Dparindent    % $B2~CJMn;~$K%$%s%G%s%H$9$k(B
%\Dnoparindent  % $B2~CJMn;~$K%$%s%G%s%H$7$J$$(B

%% Macro defined by author
\def\univec#1{ \hat{ \Dvect{\rm #1}} }
\def\DD#1#2{\frac{\mathrm D #1}{\mathrm D #2}}
\def\dd#1#2{\frac{\mathrm d #1}{\mathrm d #2}}
\def\Dd#1{\; {\mathrm d} #1}
\def\D#1{\Dvect{#1}}
\def\p{\prime}
\renewcommand{\DP}[3][]{\frac{\partial^{#1} #2}{\partial #3^{#1}}}
\def\ol#1{\overline{#1}}
\def\wh#1{\widehat{#1}}
\def\wt#1{\widetilde{#1}}
\hypersetup{
  colorlinks,
  citecolor=red,
  linkcolor=blue,
  urlcolor=blue
}
\bibpunct{(}{)}{,}{a}{,}{,}

\begin{document}
%
$BBjL\(B: $BMk1@$rA[Dj$7$?6/@)$K$h$j@8$8$k(B
$B5pBgOG@1I=AXN.$N?tCM7W;;(B
%
%
\begin{center}
  \section*{$BMW;](B}
\end{center}
%% Abstract %%
%%%%%%%%%%%%%%%%
%%
\quad 
%%%
$B5pBgOG@1I=AX$K$O!$(B
$B@VF;0h$K$*$1$k0^EYI}$N9-$$%8%'%C%H$d(B
$BCf0^EY0h$K$*$1$k0^EYI}$N69$$%8%'%C%H$+$i@.$kBS>u9=B$!$(B
$B6K0h$K$*$1$k129=B$$,B8:_$9$k$3$H$,CN$i$l$F$$$k!%(B
%%%
%
$BBS>u9=B$$d6K12$N$h$&$J(B
$BBg5,LO9=B$$O%(%M%k%.!<$N5U%+%9%1!<%I8z2L(B(\cite{Vallis2017})$B$K$h$C$F!$(B
$BBg5$Cf$N>.5,LOMpN.$+$i7A@.$5$l$k2DG=@-$,$"$k!%(B
%
$B$3$N$h$&$J>.5,LOMpN.$r(B
$B0z$-5/$3$98uJd$H$7$F!$(B
$BMk1@$,9M$($i$l$k(B($BNc$($P!$(B\cite{Ingersoll2000})$B!%(B
%
\cite{Showman2007}, \cite{Brueshaber2019} $B$O(B
$B$3$NMk1@$rA[Dj$7$?<ANL6/@)$r2C$($?@u?e<B83$r(B
$B5eLL$N0lIt$NNN0h$G9T$C$?!%(B
%
\cite{Showman2007} $B$O0^EY(B$0^\circ - 70^\circ$$B!$7PEY(B$0^\circ - 120^\circ$ $B$NNN0h$r7W;;$7$?7k2L!$(B
$BBS>u9=B$$,7A@.$5$l$k$3$H$r<($7$?!%(B
%
\cite{Brueshaber2019} $B$O0^EY(B$60^\circ$ $B$h$j9b0^EY$N6K0h$r7W;;$7!$(B
$B6K12$,7A@.$5$l$k$3$H$r<($7$?!%(B
$B$=$7$F!$6K12$N?t$dBg$-$5!$12EY$NId9f$H$$$C$?FCD'$O(B
Burger $B?t(B($BJQ7AH>7B$HOG@1H>7B$NHf$G=q$+$l$kL5<!85NL(B) $B$NCM$K(B
$B6/$/0MB8$9$k7k2L$rF@$?!%(B
%%%%%%%%%%%%%%%%%%%%%%%%%%%%%%%%%%%%%%%%%%%%%%%%% parallel
%
% $BCJMn$r$+$($F!$K\8&5f$NL\E*$O(B... $B$H(B1$B$D$NCJMn$K(B

$B$7$+$7!$(B\cite{Showman2007}, \cite{Brueshaber2019}$B!!$O$I$A$i$bNN0h7W;;$G$"$C$?!%(B
$B$=$N$?$a!$A45e7W;;$r9T$$!$H`$i$N7k2L$r3N$+$a$kI,MW$,$"$k!%(B
%
$B$J$<$J$i!$@h9T8&5f$G$O9MN8$5$l$F$$$J$$(B
$B7W;;NN0h30$+$i$N1?F0NLM"Aw$J$I$K$h$j!$(B
$B7W;;$GF@$i$l$?%8%'%C%H$d(B
$B12$N9=B$$,JQ2=$9$k2DG=@-$,$"$k$+$i$G$"$k!%(B
%
%
$BK\8&5f$NL\E*$OMk1@$rA[Dj$7$FM?$($?<ANL6/@)$K$h$k(B
$B?tCM<B83$rA45e$G9T$$!$Mk1@$K$h$k6/@)$K$h$j!$(B
$B5pBgOG@1$G8+$i$l$kBS>u9=B$$H6K12$,7A@.$5$l$k$N$+$rD4$Y$k$3$H$G$"$k!%(B
$B2C$($F!$$=$l$i$N9=B$$r@h9T8&5f$GF@$i$l$?7k2L$HHf3S$7!$(B
$BNN0h7W;;$GF@$i$l$?7k2L$NBEEv@-$rD4$Y$k!%(B

$BK\8&5f$OCO5eN.BNEEG>6f3ZIt$N3,AXE*CO5e%9%Z%/%H%k%b%G%k=8(B
(SPMODEL; \cite{spmodel2006}, \cite{spmodel2013})$B$rMQ$$!$(B
$BMk1@$rA[Dj$7$?<ANL6/@)$r2C$($?(B1.5$BAX@u?e7OA45e7W;;$r9T$C$?!%(B
%
$B7W;;$N7k2L!$A45e7W;;$G$b(B
\cite{Showman2007} $B$GF@$i$l$?BS>u9=B$$H(B
\cite{Brueshaber2019} $B$GF@$i$l$?129=B$$,7A@.$9$k$H$$$&7k2L$,F@$i$l$?!%(B
% jet
$B$^$?!$BS>u9=B$$d6K12$N9=B$$O(B
$B<ANL6/@)$N@5Ii$N3d9g$d!$(B
$B6u4VE*Bg$-$5$rJQ99$7$F$bJQ$o$i$:!$(B
$B6K12$NFCD'$O(BBurger $B?t$NCM$K6/$/0MB8$9$k7k2L$rF@$?!%(B
%   $B6qBNE*$J12$N>uBV(B
Burger $B?t$,>.$5$$>l9g!$LZ@1$G4QB,$5$l$k$h$&$J(B
$BHf3SE*>.$5$JJ#?t$N12$,7A@.$7!$(B
Burger $B?t$,Bg$-$/$J$k$K$D$l$F!$(B
$BEZ@1!$E72&@1!$3$2&@1$G(B
$B4QB,$5$l$F$$$kC10l$NDc5$05@-12$,7A@.$9$k!%(B
%
%
$B$3$l$i$N7k2L$O!$(B\cite{Showman2007}, \cite{Brueshaber2019} $B$G(B
$BF@$i$l$?7k2L$H@09gE*$G$"$k!%(B
%
$B$?$@$7!$:#2s$N7W;;7k2L$G$O@VF;0h$G$N%8%'%C%H$NIwB.$,(B\cite{Showman2007} $B$N7k2L$KHf$Y$F!$(B
$BLs(B10$BG\Bg$-$$!%$3$N7k2L$O(B\cite{Showman2007} $B$G$O9MN8$5$l$F$$$J$$(B
$B6K0h$"$k$$$OH?BPH>5e$NB8:_$,%8%'%C%H$N6/EY$K1F6A$rM?$($k2DG=@-$r<(:6$7$F$$$k$+$b$7$l$J$$!%(B
% vor
%
%
                  
%
\newpage

Title : Numerical experiments of giant planet surface flows 
produced by the forcing representing thunderstorms
%
\\
\\
\quad 
%%%
It is known that the surface of giant planets have
the banded structure consisting of jets with a wide latitude in the equatorial region, 
jets with a narrow latitude in the mid-latitude region, 
and the vortex structure in the polar region.
%$B5pBgOG@1I=AX$K$O!$(B
%$B@VF;0h$K$*$1$k0^EYI}$N9-$$%8%'%C%H$d(B
%$BCf0^EY0h$K$*$1$k0^EYI}$N69$$%8%'%C%H$+$i@.$kBS>u9=B$!$(B
%$B6K0h$K$*$1$k129=B$$,B8:_$9$k$3$H$,CN$i$l$F$$$k!%(B
%
Large-scale structures such as bands and polar vortices 
can be formed from small-scale turbulence in the atmosphere 
due to the inverse energy cascade effect(\cite{Vallis2017}). 
Thunderstorms are considered to be 
a candidate for causing such small-scale turbulence(e.g. \cite{Ingersoll2000}).
%$BBS>u9=B$$d6K12$N$h$&$J(B  
%$BBg5,LO9=B$$O%(%M%k%.!<$N5U%+%9%1!<%I8z2L(B(\cite{Vallis2017})$B$K$h$C$F!$(B
%$BBg5$Cf$N>.5,LOMpN.$+$i7A@.$5$l$k2DG=@-$,$"$k!%(B
%$B$3$N$h$&$J>.5,LOMpN.$r(B
%$B0z$-5/$3$98uJd$H$7$F!$(B
%$BMk1@$,9M$($i$l$k(B(\cite{Gierasch2000}, \cite{Ingersoll2000})$B!%(B 
%% 
%
\cite{Showman2007} and \cite{Brueshaber2019} calculated 
shallow water experiment with the mass forcing representing thunderstorms 
in a part of the sphere.
%\cite{Showman2007}, \cite{Brueshaber2019} $B$O(B
%$B$3$NMk1@$rA[Dj$7$?<ANL6/@)$r2C$($?@u?e<B83$r(B
%$B5eLL$N0lIt$NNN0h$G9T$C$?!%(B
%
%
\cite{Showman2007} calculated the area of latitude $0^\circ - 70^\circ$ and
longitude $0^\circ - 120^\circ$$B!$(B
Their results showed the formation of the zonal banded structure.
%\cite{Showman2007} $B$O0^EY(B$0^\circ - 70^\circ$$B!$7PEY(B$0^\circ - 120^\circ$ $B$NNN0h$r7W;;$7$?7k2L!$(B
%$BBS>u9=B$$,7A@.$5$l$k$3$H$,$o$+$C$?!%(B
%
\cite{Brueshaber2019} calculated the polar region at latitudes higher than $60^\circ$ 
and the polar vortices were formed.
%\cite{Brueshaber2019} $B$O0^EY(B$60^\circ$ $B$h$j9b0^EY$N6K0h$r7W;;$7!$(B
%$B6K12$,7A@.$5$l$k$3$H$,$o$+$C$?!%(B
%
In addition, ther calculation results showed 
that features such as the number and size of polar vortices,
cyclones and anticylones largely depend on 
the value of the Burger number (dimensionless number written 
by the ratio of the deformation radius to the planetary radius).
%$B$^$?!$6K12$N?t$dBg$-$5!$Dc5$05@-!&9b5$05@-$H$$$C$?FCD'$O(B
%Burger $B?t(B($BJQ7AH>7B$HOG@1H>7B$NHf$G=q$+$l$kL5<!85NL(B) $B$NCM$K(B
%$B6/$/0MB8$9$k$3$H$,$o$+$C$?!%(B

However, both previous studies are area calculations, 
and it is necessary to perform global calculations and confirm their results.
%$B$7$+$7!$$I$A$i$N@h9T8&5f$bNN0h7W;;$G$"$j!$(B
%$BA45e7W;;$r9T$$!$H`$i$N7k2L$r3N$+$a$kI,MW$,$"$k!%(B
%
This is because the calculated jet and vortex structures 
may change due to momentum transport from outside the computational domain, 
which was not considered in the previous studies.
%$B$J$<$J$i!$@h9T8&5f$G$O9MN8$5$l$F$$$J$$(B
%$B7W;;NN0h30$+$i$N1?F0NLM"Aw$J$I$K$h$j!$(B
%$B7W;;$GF@$i$l$?%8%'%C%H$d(B
%$B12$N9=B$$,JQ2=$9$k2DG=@-$,$"$k$+$i$G$"$k!%(B
%
The purpose of this study is to investigate
wheter the jet and polar vortex structures seen on giant planets are
formed by the mass forcing representing thunderstorms.
%$BK\8&5f$NL\E*$OMk1@$rA[Dj$7$FM?$($?<ANL6/@)$K$h$k(B
%$B?tCM<B83$rA45e$G9T$$!$Mk1@$K$h$k6/@)$K$h$j!$(B
%$B5pBgOG@1$G8+$i$l$kBS>u9=B$$H6K12$,7A@.$5$l$k$N$+$rD4$Y$k$3$H$G$"$k(B
%
Then, compare my global calculation results with the results of previous studies 
and consider the validity of the results obtained by domain calculation.
%$B2C$($F!$$=$l$i$N9=B$$r@h9T8&5f$GF@$i$l$?7k2L$HHf3S$7!$(B
%$BNN0h7W;;$GF@$i$l$?7k2L$NBEEv@-$rD4$Y$k!%(B

This study used Hierarchical Spectral Models for GFD (SPMODEL; \cite{spmodel2006}, \cite{spmodel2013}).
%
The equation system is a 1.5-layer shallow water system
with the mass forcing representing thunderstorms.
And the calculation are is global.
%$BK\8&5f$OCO5eN.BNEEG>6f3ZIt$N3,AXE*CO5e%9%Z%/%H%k%b%G%k=8(B
%(SPMODEL; \cite{spmodel2006}, \cite{spmodel2013})$B$rMQ$$!$(B
%$BMk1@$rA[Dj$7$?<ANL6/@)$r2C$($?(B1.5$BAX@u?e7OA45e7W;;$r9T$&!%(B
%
The results of the experiments showed 
that the banded structure obtained by \cite{Showman2007} and 
the polar vortex structure obtatine by \cite{Brueshaber2019} were formed.
%$B7W;;$N7k2L!$A45e7W;;$G$b(B
%\cite{Showman2007} $B$GF@$i$l$?BS>u9=B$$H(B
%\cite{Brueshaber2019} $B$GF@$i$l$?129=B$$,7A@.$9$k$H$$$&7k2L$,F@$i$l$?!%(B
%
It was also found that the structure does not chenge even if the positive / negative 
ratio of the mass forcing and the spatial size are changed.
%$B$^$?!$BS>u9=B$$d6K12$O<ANL6/@)$N@5Ii$N3d9g$d!$(B
%$B6u4VE*Bg$-$5$rJQ99$7$F$bJQ$o$i$J$$$3$H$,$o$+$C$?!%(B
%
The characteristics of the polar vortex were 
found to depend on the value of the Burger number:
when the Burger number is small, 
relatively small multiple vortices are formed, as observed Jupiter,
and as the Burger number increases, 
a single cyclonic vortex is formed, observed on Saturn, Uranus, and Neptune.
%$B$^$?!$6K12$NFCD'$O(BBurger $B?t$NCM$K0MB8$9$k$3$H$,$o$+$C$?!%(B
%Burger $B?t$,>.$5$$>l9g!$LZ@1$G4QB,$5$l$k$h$&$J(B
%$BHf3SE*>.$5$JJ#?t$N12$,7A@.$7!$(B
%Burger $B?t$,Bg$-$/$J$k$K$D$l$F!$(B
%$BEZ@1!$E72&@1!$3$2&@1$G(B
%$B4QB,$5$l$F$$$kC10l$NDc5$05@-12$,7A@.$9$k!%(B
%
These results are consistent with the results of \cite{Showman2007} and \cite{Brueshaber2019}.
%$B$3$l$i$N7k2L$O!$(B\cite{Showman2007}, \cite{Brueshaber2019} $B$G(B
%$BF@$i$l$?7k2L$H@09gE*$G$"$k!%(B
%
However, my calculation results show that 
the zonal wind speed of the equatorial jet is 
about 10 times larger than that of \cite{Showman2007}.
%$B$7$+$7!$:#2s$N7W;;7k2L$G$O@VF;0h$G$N%8%'%C%H$NIwB.$,(B\cite{Showman2007} $B$N7k2L$KHf$Y$F!$(B
%$BLs(B10$BG\Bg$-$/$J$k$3$H$,$o$+$C$?!%(B
%
The result suggests that the polar reion or opposite hemisphere,
which are not considered in \cite{Showman2007}, may affect the strength of the jet.
%$B$3$N7k2L$O(B\cite{Showman2007} $B$G$O9MN8$5$l$F$$$J$$(B
%$B6K0h$"$k$$$OH?BPH>5e$NB8:_$,(B
%$B%8%'%C%H$N6/EY$K1F6A$rM?$($k2DG=@-$r<(:6$7$F$$$k!%(B
%
\bibliographystyle{abbrvnat}
\bibliography{./reference/reference}

\end{document}
